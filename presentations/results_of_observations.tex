\documentclass[a4paper, 12pt]{article}

\usepackage{cmap} 
\usepackage[T2A]{fontenc} 
\usepackage[utf8]{inputenc} 
\usepackage[english, russian]{babel}

%%% Математика 
\usepackage{amssymb} \usepackage{dsfont}

%%% Поля документа 
\usepackage{geometry} 
\geometry{top = 10mm} 
\geometry{bottom = 20mm} 
\geometry{left = 20mm}
\geometry{right = 20mm}

%%% Красная строка 
\usepackage{indentfirst}

%%% Заголовок 
\author{Сухочев Александр и Балакший Андрей} 
\title{Результаты наблюдений} 

%%% Теоремы 
\usepackage{amsthm} 
\theoremstyle{plain} 
\newtheorem{proposition}{Утверждение}[section]

\theoremstyle{definition} 
\newtheorem{theorem}{Теорема}[proposition]

\theoremstyle{remark} 
\newtheorem{remarks}{Следствие}[section]

%%% Начало документа
\begin{document}
\maketitle

Гляньте на эту табличку!

Logistic Regression with 1 0

Accuracy: 0.663265306122

All test count: 98; TP: 7; TN: 58; FP: 10; FN: 23

и

Logistic Regression with tf idf

Accuracy: 0.704081632653

All test count: 98; TP: 1; TN: 68; FP: 0; FN: 29

С Тф идф выдает лучший результат, но посмотрите на ТП! Просто так совпало что у нас больше негативных в тестирующем множестве, поэтому мы не можем сказать что тф идф лучше работает, видно ведь что он хуже определяет Позитивные! Переделывать обучающее множество нельзя, т.е. у нас и при работе программы будет подобное соотношение позитивных/негативных, но спасибо за полезный опыт

\section{Таблицы точностей различных машинных обучений и экстракторов.}

\subsection{Общая таблица}
\begin{tabular} {|c|c|c|c|c|}
\hline
   ~~~~ & SE & N(only)-gramm & N-gramm E & SE with not \\
   \hline
  MNB & 0.6655 & & & \\
   \hline
  GNB & 0.6224 & & & \\
   \hline
  SVM 1 0 & 0.6834 & & & \\
   \hline
  SVM tf-idf & 0.6789 & & & \\
   \hline
  LogReg & 0.7013 & & & \\
\hline
\end{tabular}

\subsection{Более подробно}
\subsubsection{Standart Extractor with mystem}
\begin{tabular}{|c|c|c|c|c|c|c|c|}
\hline
  ~~~~ & Correct & Total & Acc & TP & TN & FP & FN \\
  \hline
  MNB & 742 & 1115 & 0.6655 & 61 & 618 & 51 & 322  \\
  \hline
  GNB & 694 & 1115 & 0.6224 & 159 & 535 & 197 & 224 \\
  \hline
  SVM 1 0 & 762 & 1115 & 0.6834 & 160 & 602 & 130 & 223 \\
  \hline
  SVM tf-idf & 757 & 1115 & 0.6789 & 71 & 686 & 46 & 312 \\
  \hline
  Log Reg count & 782 & 1115 & 0.7013 & 134 & 648 & 84 & 249 \\
\hline
\end{tabular}


\subsubsection{Standart Extractor without mystem}
\begin{tabular}{|c|c|c|c|c|c|c|c|}
\hline
  ~~~~ & Correct & Total & Acc & TP & TN & FP & FN \\
  \hline
  MNB & 668 & 1115 & 0.5991 & 149 & 519 & 213 & 234  \\
  \hline
  GNB & 694 & 1115 & 0.6224 & 159 & 535 & 197 & 224 \\
  \hline
  SVM 1 0 & 729 & 1115 & 0.6538 & 120 & 609 & 123 & 263 \\
  \hline
  SVM tf-idf & 757 & 1115 & 0.6789 & 48 & 709 & 23 & 335 \\
  \hline
  Log Reg count & 744 & 1115 & 0.6673 & 96 & 648 & 84 & 287 \\
\hline
\end{tabular}

Вывод: гляньте как плохо всё без майстема! Вывод: а что поменялось ?

\section{Фичи}
\subsection{Standard Extractor with mystem}
\subsubsection{Склеивать не + слово}



\subsubsection{Количество строк}
SVM 1 0

\begin{tabular}{|c|c|c|c|c|c|c|c|}
\hline
  ~~~~ & Correct & Total & Acc & TP & TN & FP & FN \\
  \hline
  SE & 762 & 1115 & 0.6834 & 160 & 602 & 130 & 223  \\
  \hline
  0 & 754 & 1115 & 0.6762 & 156 & 598 & 134 & 227  \\
  \hline
  1 & 754 & 1115 & 0.6762 & 156 & 598 & 134 & 227  \\
  \hline
  2 & 694 & 1115 & 0.6224 & 159 & 535 & 197 & 224 \\
  \hline
  3 & 729 & 1115 & 0.6538 & 120 & 609 & 123 & 263 \\
  \hline
  4 & 757 & 1115 & 0.6789 & 48 & 709 & 23 & 335 \\
  \hline
  5 & 744 & 1115 & 0.6673 & 96 & 648 & 84 & 287 \\
\hline
\end{tabular}

\end{document}