\documentclass[t]{beamer}
\usetheme{Szeged}
\usecolortheme{beaver}
\usepackage{cmap}					%поиск в pdf
\usepackage[T2A]{fontenc}			% кодировка
\usepackage[utf8]{inputenc}			% кодировка исходного текста
\usepackage[english,russian]{babel}	% локализация и переносы
%%% Работа с картинками
\usepackage{graphicx}  % Для вставки рисунков
\graphicspath{{images/}{images2/}}  % папки с картинками
\setlength\fboxsep{3pt} % Отступ рамки \fbox{} от рисунка
\setlength\fboxrule{1pt} % Толщина линий рамки \fbox{}
\usepackage{wrapfig} % Обтекание рисунков текстом

%%% Работа с таблицами
\usepackage{array,tabularx,tabulary,booktabs} % Дополнительная работа с таблицами
\usepackage{longtable}  % Длинные таблицы
\usepackage{multirow} % Слияние строк в таблице

%%% Программирование
\usepackage{etoolbox} % логические операторы

%%% Другие пакеты
\usepackage{lastpage} % Узнать, сколько всего страниц в документе.
\usepackage{soul} % Модификаторы начертания
\usepackage{csquotes} % Еще инструменты для ссылок
%\usepackage[style=authoryear,maxcitenames=2,backend=biber,sorting=nty]{biblatex}
\usepackage{multicol} % Несколько колонок

%%% Картинки
\usepackage{tikz} % Работа с графикой
\usepackage{pgfplots}
\usepackage{pgfplotstable}

\title{Senty}
\author{Балакший Андрей \and Сухочев Александр 
	\and \newline Куратор:Юрий Курочкин}
\date{19 мая 2015}
\institute[Computer Science Center]

\begin{document}
	\frame[plain]{\titlepage}
	
	
	\section{О проекте}
	\begin{frame}
		\frametitle{\insertsection}
		\textbf{Поиск упоминаний и сентиментальная разметка коротких и 	очень коротких текстов.(с использованием "читой" разметки обучающего множества)}
	\end{frame}
	
	
	\section{Вдохновение}
	\begin{frame}
		\frametitle{\insertsection}
		\textbf{Прочитали статью студентов из Стэнфорда о сентиментальной разметке твитов. Ребята производили "грязную" разметку обучающего множества и обучались по ней.}\pause
		
		\textbf{Захотелось чего-то аналогичного, но с русским языком и с "чистой" разметкой.}
	\end{frame}
	
	
	\section{Цель}
	\begin{frame}
		\frametitle{\insertsection}
		\textbf{Хотим понимать настроение и эмоции человека по написанному им русскоязычному тексту.}
	\end{frame}
	
	
	\section{Что хотим сделать}
	\begin{frame}
		\frametitle{\insertsection}
		\begin{itemize}
			\item Найти источник для обучения, содержащий высказывания людей по тем или иным темам.	
			\item Придумать различные фичи для текстов из нашего источника(признаки, по которым можно определять настроения людей, написавших данный текст) и реализовать по ним машинное обучение.
		\end{itemize}
	\end{frame}
	
	
	\section{Этапы выполнения}
	\subsection{Получение материала для обучения}
	
	
	
	
	
	
\end{document}